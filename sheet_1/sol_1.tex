%%%%%%%%%%%%%%%%%%%%%%%%%%%%%%%%%%%%%%%%%%%%%%%%%%%%%%%%%%%%%%%%%%%%%%%%%%
%Author:																 %
%-------																 %
%Yannis Baehni at University of Zurich									 %
%baehni.yannis@uzh.ch													 %
%																		 %
%Version log:															 %
%------------															 %
%06/02/16 . Basic structure												 %
%04/08/16 . Layout changes including section, contents, abstract.		 %
%%%%%%%%%%%%%%%%%%%%%%%%%%%%%%%%%%%%%%%%%%%%%%%%%%%%%%%%%%%%%%%%%%%%%%%%%%

%Page Setup
\documentclass[
	11pt, 
	oneside, 
	a4paper,
	reqno,
	final
]{amsart}

\usepackage[
	left = 3cm, 
	right = 3cm, 
	top = 3cm, 
	bottom = 3cm
]{geometry}

%Headers and footers
\usepackage{fancyhdr}
	\pagestyle{fancy}
	%Clear fields
	\fancyhf{}
	%Header right
	\fancyhead[R]{
		\footnotesize
		Yannis B\"{a}hni\\
		\href{mailto:yannis.baehni@uzh.ch}{yannis.baehni@uzh.ch}
	}
	%Header left
	\fancyhead[L]{
		\footnotesize
		MAT901: Stochastics I\\
		Spring Semester 2017
	}
	%Page numbering in footer
	\fancyfoot[C]{\thepage}
	%Separation line header and footer
	\renewcommand{\headrulewidth}{0.4pt}
	%\renewcommand{\footrulewidth}{0.4pt}
	
	\setlength{\headheight}{19pt} 

%Title
\usepackage[foot]{amsaddr}
\usepackage{mathrsfs}
%\usepackage{mathptmx}
\usepackage{xspace}
\makeatletter
\def\@textbottom{\vskip \z@ \@plus 1pt}
\let\@texttop\relax
\usepackage{etoolbox}
\patchcmd{\abstract}{\scshape\abstractname}{\textbf{\abstractname}}{}{}

\usepackage[all,cmtip]{xy}

%Switching commands for different section formats
%Mainsectionsytle
\newcommand{\mainsectionstyle}{%
  	\renewcommand{\@secnumfont}{\bfseries}
  	\renewcommand\section{\@startsection{section}{1}%
    	\z@{.5\linespacing\@plus.7\linespacing}{-.5em}%
    	{\normalfont\bfseries}}%
	\renewcommand\subsection{\@startsection{subsection}{2}%
    	\z@{.5\linespacing\@plus.7\linespacing}{-.5em}%
    	{\normalfont\bfseries}}%
	\renewcommand\subsubsection{\@startsection{subsubsection}{3}%
    	\z@{.5\linespacing\@plus.7\linespacing}{-.5em}%
    	{\normalfont\bfseries}}%
}
\newcommand{\originalsectionstyle}{%
\def\@secnumfont{\bfseries}%\mdseries
\def\section{\@startsection{section}{1}%
  \z@{.7\linespacing\@plus\linespacing}{.5\linespacing}%
  {\normalfont\bfseries\centering}}
}
%Formatting title of TOC
\renewcommand{\contentsnamefont}{\bfseries}
%Table of Contents
\setcounter{tocdepth}{3}

% Add bold to \section titles in ToC and remove . after numbers
\renewcommand{\tocsection}[3]{%
  \indentlabel{\@ifnotempty{#2}{\bfseries\ignorespaces#1 #2\quad}}\bfseries#3}
% Remove . after numbers in \subsection
\renewcommand{\tocsubsection}[3]{%
  \indentlabel{\@ifnotempty{#2}{\ignorespaces#1 #2\quad}}#3}
\let\tocsubsubsection\tocsubsection% Update for \subsubsection
%...

\newcommand\@dotsep{4.5}
\def\@tocline#1#2#3#4#5#6#7{\relax
  \ifnum #1>\c@tocdepth % then omit
  \else
    \par \addpenalty\@secpenalty\addvspace{#2}%
    \begingroup \hyphenpenalty\@M
    \@ifempty{#4}{%
      \@tempdima\csname r@tocindent\number#1\endcsname\relax
    }{%
      \@tempdima#4\relax
    }%
    \parindent\z@ \leftskip#3\relax \advance\leftskip\@tempdima\relax
    \rightskip\@pnumwidth plus1em \parfillskip-\@pnumwidth
    #5\leavevmode\hskip-\@tempdima{#6}\nobreak
    \leaders\hbox{$\m@th\mkern \@dotsep mu\hbox{.}\mkern \@dotsep mu$}\hfill
    \nobreak
    \hbox to\@pnumwidth{\@tocpagenum{\ifnum#1=1\bfseries\fi#7}}\par% <-- \bfseries for \section page
    \nobreak
    \endgroup
  \fi}
\AtBeginDocument{%
\expandafter\renewcommand\csname r@tocindent0\endcsname{0pt}
}
\def\l@subsection{\@tocline{2}{0pt}{2.5pc}{5pc}{}}
\def\l@subsubsection{\@tocline{2}{0pt}{4.5pc}{5pc}{}}
\makeatother

\advance\footskip0.4cm
\textheight=54pc    %a4paper
\textheight=50.5pc %letterpaper
\advance\textheight-0.4cm
\calclayout

%Font settings
%\usepackage{anyfontsize}
%Footnote settings
%\usepackage{mathptmx}
\usepackage{footmisc}
%	\renewcommand*{\thefootnote}{\fnsymbol{footnote}}
\usepackage{commath}
%Further math environments
%Further math fonts (loads amsfonts implicitely)
\usepackage{amssymb}
%Redefinition of \text
%\usepackage{amstext}
\usepackage{upref}
%Graphics
%\usepackage{graphicx}
%\usepackage{caption}
%\usepackage{subcaption}
%Frames
\usepackage{mdframed}
\allowdisplaybreaks
%\usepackage{interval}
\newcommand{\toup}{%
  \mathrel{\nonscript\mkern-1.2mu\mkern1.2mu{\uparrow}}%
}
\newcommand{\todown}{%
  \mathrel{\nonscript\mkern-1.2mu\mkern1.2mu{\downarrow}}%
}
\AtBeginDocument{\renewcommand*\d{\mathop{}\!\mathrm{d}}}
\renewcommand{\Re}{\operatorname{Re}}
\renewcommand{\Im}{\operatorname{Im}}
\DeclareMathOperator\Log{Log}
\DeclareMathOperator\Arg{Arg}
\DeclareMathOperator\sech{sech}
\DeclareMathOperator*\esssup{ess.sup}
\DeclareMathOperator\id{id}
%\usepackage{hhline}
%\usepackage{booktabs} 
%\usepackage{array}
%\usepackage{xfrac} 
%\everymath{\displaystyle}
%Enumerate
\usepackage{tikz}
\usetikzlibrary{external}
\tikzexternalize % activate!
\usetikzlibrary{patterns}
\pgfdeclarepatternformonly{adjusted lines}{\pgfqpoint{-1pt}{-1pt}}{\pgfqpoint{40pt}{40pt}}{\pgfqpoint{39pt}{39pt}}%
{
  \pgfsetlinewidth{.8pt}
  \pgfpathmoveto{\pgfqpoint{0pt}{0pt}}
  \pgfpathlineto{\pgfqpoint{39.1pt}{39.1pt}}
  \pgfusepath{stroke}
}
\usepackage{enumitem} 
%\renewcommand{\labelitemi}{$\bullet$}
%\renewcommand{\labelitemii}{$\ast$}
%\renewcommand{\labelitemiii}{$\cdot$}
%\renewcommand{\labelitemiv}{$\circ$}
%Colors
%\usepackage{color}
%\usepackage[cmtip, all]{xy}
%Theorems
\newtheoremstyle{bold}              	 %Name
  {}                                     %Space above
  {}                                     %Space below
  {\itshape}		                     %Body font
  {}                                     %Indent amount
  {\bfseries}                             %Theorem head font
  {.}                                    %Punctuation after theorem head
  { }                                    %Space after theorem head, ' ', 
  										 %	or \newline
  {\thmname{#1}\thmnumber{ #2}\thmnote{ (#3)}} 
\theoremstyle{bold}
\newtheorem*{definition*}{Definition}
\newtheorem{definition}{Definition}[section]
\newtheorem*{lemma*}{Lemma}
\newtheorem{lemma}{Lemma}[section]
\newtheorem{Proof}{Proof}[section]
\newtheorem{proposition}{Proposition}[section]
\newtheorem{properties}{Properties}[section]
\newtheorem{corollary}{Corollary}[section]
\newtheorem*{theorem*}{Theorem}
\newtheorem{theorem}{Theorem}[section]
\newtheorem{example}{Example}[section]
\newtheoremstyle{nonitalic}            	 %Name
  {}                                     %Space above
  {}                                     %Space below
  {}				                     %Body font
  {}                                     %Indent amount
  {\bfseries}                             %Theorem head font
  {.}                                    %Punctuation after theorem head
  { }                                    %Space after theorem head, ' ', 
  										 %	or \newline
  {\thmname{#1}\thmnumber{ #2}\thmnote{ (#3)}}
\theoremstyle{nonitalic}
\newtheorem*{remark*}{Remark}
\newtheorem{remark}{Remark}[section]
%German non-ASCII-Characters
%Graphics-Tool
%\usepackage{tikz}
%\usepackage{tikzscale}
%\usepackage{bbm}
%\usepackage{bera}
%Listing-Setup
%Bibliographie
\usepackage[backend=bibtex, style=alphabetic]{biblatex}
%\usepackage[babel, german = swiss]{csquotes}
\bibliography{Bibliography}
%PDF-Linking
%\usepackage[hyphens]{url}
\usepackage[bookmarksopen=true,bookmarksnumbered=true]{hyperref}
%\PassOptionsToPackage{hyphens}{url}\usepackage{hyperref}
\hypersetup{
  colorlinks   = true, %Colours links instead of ugly boxes
  urlcolor     = blue, %Colour for external hyperlinks
  linkcolor    = blue, %Colour of internal links
  citecolor    = blue %Colour of citations
}
%Weierstrass-P symbol for power set
\newcommand{\powerset}{\raisebox{.15\baselineskip}{\Large\ensuremath{\wp}}}
\newcommand{\bld}[1]{\boldmath\textit{\textbf{#1}}\unboldmath}


\title{Solutions Sheet 1}
\author{Yannis B\"{a}hni}
\address[Yannis B\"{a}hni]{University of Zurich, R\"{a}mistrasse 71, 8006 Zurich}
\email[Yannis B\"{a}hni]{\href{mailto:yannis.baehni@uzh.ch}{yannis.baehni@uzh.ch}}

\begin{document}
\maketitle
\thispagestyle{fancy}
\begin{enumerate}[label = \textbf{Exercise \arabic*.},wide = 0pt, itemsep=1.5ex]
\item
	First we label the balls: the green ones with $\cbr[0]{1,\dots,17}$, the blue ones with $\cbr[0]{18,\dots,22}$ and the red ones with $\cbr[0]{23,\dots,33}$.
	\begin{enumerate}[label = \arabic*.,wide = 0pt, itemsep=1.5ex]
		\item Define the sample space as
			\begin{equation*}
				\Omega := \cbr[0]{\omega : \omega = (a_1,a_2), a_j \neq a_k, j\neq k, a_i \in \cbr[0]{1,\dots,33}}.
			\end{equation*}

			Assume the outcomes are equally probable. Then 
			\begin{equation*}
				P(A) = \frac{\abs[0]{A}}{\abs[0]{\Omega}} = \frac{\abs[0]{A}}{33 \cdot 32}
			\end{equation*}

			\noindent for any $A \in 2^\Omega$. Therefore
			\begin{equation*}
				P(\cbr[0]{\text{none red}}) = \frac{\abs[0]{\cbr[0]{\omega : \omega = (a_1,a_2), a_j \neq a_k, j\neq k, a_i \in \cbr[0]{1,\dots,22}}}}{33 \cdot 32}= \frac{22 \cdot 21}{33 \cdot 32}= \frac{7}{16}
			\end{equation*}

		\item Define the sample space as 
			\begin{equation*}
				\Omega := \cbr[0]{\omega : \omega = (a_1,a_2,a_3), a_i \in \cbr[0]{1,\dots,33}} \qquad \abs[0]{\Omega} = 33^3.
			\end{equation*}

			Now we have 
			
			\begin{equation*}
				P(\cbr[0]{\text{at most two green}}) = 1 - P(\cbr[0]{\text{exactly three green}})
			\end{equation*}

			\noindent and so by 
			\begin{equation*}
				P(\cbr[0]{\text{exactly three green}}) = \del[3]{\frac{17}{33}}^3
			\end{equation*}

			\noindent we get 
			\begin{equation*}
				P(\cbr[0]{\text{at most two green}}) = 1 - \del[3]{\frac{17}{33}}^3
			\end{equation*}
	\end{enumerate}
\item
	This can be modeled by sampling without replacement if we draw one card after the other. Therefore we consider the sample space
		\begin{equation*}
				\Omega := \cbr[0]{\omega : \omega = (a_1,a_2,a_3,a_4,a_5), a_j \neq a_k, j\neq k, a_i \in \cbr[0]{1,\dots,52}}.
		\end{equation*}

		Thus
		\begin{equation}
			\abs[0]{\Omega} = 52 \cdot 51 \cdot 50 \cdot 49 \cdot 48.
		\end{equation}
	\begin{enumerate}[label = \arabic*.,wide = 0pt, itemsep=1.5ex]
		\item 
			We have
			\begin{equation*}
				P(\cbr[0]{\text{same suit}}) = \frac{\abs[0]{A_1 \cup A_2 \cup A_3 \cup A_4}}{52 \cdot 51 \cdot 50 \cdot 49 \cdot 48} = \frac{4 \cdot 13 \cdot 12 \cdot 11 \cdot 10 \cdot 9}{52 \cdot 51 \cdot 50 \cdot 49 \cdot 48} = \frac{33}{16660}
			\end{equation*}

			\noindent where 
			\begin{align*}
				A_1 &:= \cbr[0]{\omega : \omega = (a_1,a_2,a_3,a_4,a_5), a_j \neq a_k, j\neq k, a_i \in \cbr[0]{1,\dots,13}}\\
				A_2 &:= \cbr[0]{\omega : \omega = (a_1,a_2,a_3,a_4,a_5), a_j \neq a_k, j\neq k, a_i \in \cbr[0]{14,\dots,26}}\\
				A_3 &:= \cbr[0]{\omega : \omega = (a_1,a_2,a_3,a_4,a_5), a_j \neq a_k, j\neq k, a_i \in \cbr[0]{27,\dots,39}}\\
				A_4 &:= \cbr[0]{\omega : \omega = (a_1,a_2,a_3,a_4,a_5), a_j \neq a_k, j\neq k, a_i \in \cbr[0]{40,\dots,52}}
			\end{align*}
	
		\item It is enough to consider the case $(a_1,a_2,a_3,a_4,a_5)$ where the first four have the same rank due to symmetry (formally we form a partition as in part a) of five subsets). For $a_1$ we have $52$ possibilities, for $a_2$ we have $3$ since there are only three cards of the same rank left, for $a_3$ we have $2$, for $a_4$ we have exactly one and finally for $a_5$ we have $48$ possibilities. Therefore we get
			\begin{equation*}
				P(\cbr[0]{\text{four cards same rank}}) = \frac{5 \cdot 52 \cdot 3 \cdot 2 \cdot 48}{52 \cdot 51 \cdot 50 \cdot 49 \cdot 48} = \frac{1}{4165}.
			\end{equation*}
		\item Again, let us consider the simplest case where the card with the lowest rank is $a_1$. Hence, for $a_1$ we have $40$ possibilities. This is due to the fact, that if we start with $10$, $11$ or $12$ we would not get an increasing sequence. For the four succeeding cards we have in total $4^4$ possibilities. Now it does not matter if we pick first the highest card or the lowest, so we have to multiply the possibilities with $S\abs[0]{S_5} = 5!$ Therefore we get
			\begin{equation*}
				P(\cbr[0]{\text{five cards sequential rank}}) = \frac{5! \cdot 40 \cdot 4^4}{52 \cdot 51 \cdot 50 \cdot 49 \cdot 48} = \frac{128}{32487}.
			\end{equation*}
		\item
			Again, we consider a simple case where the first three cards are of the same rank and then the other two follow. Therefore
			\begin{equation*}
				P(\cbr[0]{\text{full house}}) = \frac{{5 \choose 3} \cdot 52 \cdot 3 \cdot 2 \cdot 48 \cdot 3}{52 \cdot 51 \cdot 50 \cdot 49 \cdot 48} = \frac{6}{4165}.
			\end{equation*}

	\end{enumerate}
\item
	We propose that
	\begin{equation}
		\sigma(\mathcal{C}) = \cbr[0]{\cup_{\iota \in I} C_\iota : I \subseteq \cbr[0]{1,\dots,n}} =: \mathcal{A}.
	\end{equation}
	
	Firstly, $\Omega \in \mathcal{A}$ since $\Omega = \cup_{i = 1}^n C_i$. Let $(A_i)_{i \in \mathbb{N}}$ be a family in $\mathcal{A}$. Then $A_i = \cup_{\iota \in I_j} C_\iota$ for $I_j \subseteq \cbr[0]{1,\dots,n}$ and $j \in \mathbb{N}$. But then it is clearly seen that $\cup_{i \in \mathbb{N}} A_i \in \mathcal{A}$. Lastly, to show that $\mathcal{A}$ is also closed under complementation it is enough to show that $C_i^c \in \mathcal{A}$ for any $i = 1,\dots,n$ and that $\mathcal{A}$ is closed under finite intersections. Fix some $i \in \cbr[0]{1,\dots,n}$. Then it is easy to see that $C_i^c = \cup_{j \neq i}C_j \in \mathcal{A}$. Furthermore, if $A,B \in \mathcal{A}$, we have $A = \cup_{\iota \in I_1} C_\iota$ and $B = \cup_{j \in J} C_j$. Hence 
	\begin{equation*}
		A \cap B = (\cup_{\iota \in I_1} C_\iota) \cap (\cup_{j \in J} C_j) = \cup_{\iota \in I \cap J}C_\iota \in \mathcal{A}.
	\end{equation*}

	Obviously, $\mathcal{C} \subseteq \mathcal{A}$ and thus $\sigma(\mathcal{C}) \subseteq \mathcal{A}$. The inclusion $\mathcal{A} \subseteq \sigma(\mathcal{C})$ is trivial.
\item 
	~
	\begin{enumerate}[label = \arabic*.,wide = 0pt, itemsep=1.5ex]
		\item We will show 
			\begin{equation}
				-P(A \Delta B) \leq P(A) - P(B) \leq P(A \Delta B).
				\label{eq:statement}
			\end{equation}

			The first inequality is equivalent to
			\begin{equation*}
				P(B) - P(A) \leq P(A \Delta B)
			\end{equation*}

			\noindent which is easily verified by considering
			\begin{equation*}
				P(A \Delta B) = P((A \cup B) \setminus (A\cap B)) = P(A \cup B) - P(A \cap B) \geq P(B) - P(A)
			\end{equation*}

			\noindent since $B \subseteq A \cup B$ and $A \cap B \subseteq A$. Similarly the second inequality in (\ref{eq:statement}) follows from
			\begin{equation*}
				P(A \Delta B) = P((A \cup B) \setminus (A\cap B)) = P(A \cup B) - P(A \cap B) \geq P(A) - P(B)
			\end{equation*}

			\noindent since $B \subseteq A \cup B$ and $A \cap B \subseteq B$.
		\item Define a sequence 
			\begin{equation*}
				B_n := \cap_{i = 1}^n A_i
			\end{equation*}

			\noindent for $n \in \mathbb{N}$. Clearly $(B_n)_{n \in \mathbb{N}}$ is decreasing and 
			\begin{equation*}
				\cap_{n \in \mathbb{N}}B_n = \cap_{n \in \mathbb{N}} A_n.
			\end{equation*}

			Hence 
			\begin{align*}
				P(\cap_{n \in \mathbb{N}} A_n) &= P(\cap_{n \in \mathbb{N}}B_n)\\
				&= \lim_{n \to \infty} P(B_n)\\
				&= \lim_{n \to \infty} P(\cap_{i = 1}^n A_i)\\
				&= 1 - \lim_{n \to \infty} P(\cup_{i = 1}^nA_i^c)\\
				&\geq 1 - \lim_{n \to \infty} \sum_{i = 1}^n P(A_i^c)\\
				&= 1.
			\end{align*}

			\noindent by continuity from above of the probability measure $P$.
	\end{enumerate}
\end{enumerate}
%\originalsectionstyle
%\printbibliography
\end{document}
