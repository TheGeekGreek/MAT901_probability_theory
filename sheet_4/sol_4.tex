%%%%%%%%%%%%%%%%%%%%%%%%%%%%%%%%%%%%%%%%%%%%%%%%%%%%%%%%%%%%%%%%%%%%%%%%%%
%Author:																 %
%-------																 %
%Yannis Baehni at University of Zurich									 %
%baehni.yannis@uzh.ch													 %
%																		 %
%Version log:															 %
%------------															 %
%06/02/16 . Basic structure												 %
%04/08/16 . Layout changes including section, contents, abstract.		 %
%%%%%%%%%%%%%%%%%%%%%%%%%%%%%%%%%%%%%%%%%%%%%%%%%%%%%%%%%%%%%%%%%%%%%%%%%%

%Page Setup
\documentclass[
	11pt, 
	oneside, 
	a4paper,
	reqno,
	final
]{amsart}

\usepackage[
	left = 3cm, 
	right = 3cm, 
	top = 3cm, 
	bottom = 3cm
]{geometry}

%Headers and footers
\usepackage{fancyhdr}
	\pagestyle{fancy}
	%Clear fields
	\fancyhf{}
	%Header right
	\fancyhead[R]{
		\footnotesize
		Yannis B\"{a}hni\\
		\href{mailto:yannis.baehni@uzh.ch}{yannis.baehni@uzh.ch}
	}
	%Header left
	\fancyhead[L]{
		\footnotesize
		MAT901: Stochastics I\\
		Spring Semester 2017
	}
	%Page numbering in footer
	\fancyfoot[C]{\thepage}
	%Separation line header and footer
	\renewcommand{\headrulewidth}{0.4pt}
	%\renewcommand{\footrulewidth}{0.4pt}
	
	\setlength{\headheight}{19pt} 

%Title
\usepackage[foot]{amsaddr}
\usepackage{mathrsfs}
%\usepackage{mathptmx}
\usepackage{xspace}
\makeatletter
\def\@textbottom{\vskip \z@ \@plus 1pt}
\let\@texttop\relax
\usepackage{etoolbox}
\patchcmd{\abstract}{\scshape\abstractname}{\textbf{\abstractname}}{}{}

\usepackage[all,cmtip]{xy}

%Switching commands for different section formats
%Mainsectionsytle
\newcommand{\mainsectionstyle}{%
  	\renewcommand{\@secnumfont}{\bfseries}
  	\renewcommand\section{\@startsection{section}{1}%
    	\z@{.5\linespacing\@plus.7\linespacing}{-.5em}%
    	{\normalfont\bfseries}}%
	\renewcommand\subsection{\@startsection{subsection}{2}%
    	\z@{.5\linespacing\@plus.7\linespacing}{-.5em}%
    	{\normalfont\bfseries}}%
	\renewcommand\subsubsection{\@startsection{subsubsection}{3}%
    	\z@{.5\linespacing\@plus.7\linespacing}{-.5em}%
    	{\normalfont\bfseries}}%
}
\newcommand{\originalsectionstyle}{%
\def\@secnumfont{\bfseries}%\mdseries
\def\section{\@startsection{section}{1}%
  \z@{.7\linespacing\@plus\linespacing}{.5\linespacing}%
  {\normalfont\bfseries\centering}}
}
%Formatting title of TOC
\renewcommand{\contentsnamefont}{\bfseries}
%Table of Contents
\setcounter{tocdepth}{3}

% Add bold to \section titles in ToC and remove . after numbers
\renewcommand{\tocsection}[3]{%
  \indentlabel{\@ifnotempty{#2}{\bfseries\ignorespaces#1 #2\quad}}\bfseries#3}
% Remove . after numbers in \subsection
\renewcommand{\tocsubsection}[3]{%
  \indentlabel{\@ifnotempty{#2}{\ignorespaces#1 #2\quad}}#3}
\let\tocsubsubsection\tocsubsection% Update for \subsubsection
%...

\newcommand\@dotsep{4.5}
\def\@tocline#1#2#3#4#5#6#7{\relax
  \ifnum #1>\c@tocdepth % then omit
  \else
    \par \addpenalty\@secpenalty\addvspace{#2}%
    \begingroup \hyphenpenalty\@M
    \@ifempty{#4}{%
      \@tempdima\csname r@tocindent\number#1\endcsname\relax
    }{%
      \@tempdima#4\relax
    }%
    \parindent\z@ \leftskip#3\relax \advance\leftskip\@tempdima\relax
    \rightskip\@pnumwidth plus1em \parfillskip-\@pnumwidth
    #5\leavevmode\hskip-\@tempdima{#6}\nobreak
    \leaders\hbox{$\m@th\mkern \@dotsep mu\hbox{.}\mkern \@dotsep mu$}\hfill
    \nobreak
    \hbox to\@pnumwidth{\@tocpagenum{\ifnum#1=1\bfseries\fi#7}}\par% <-- \bfseries for \section page
    \nobreak
    \endgroup
  \fi}
\AtBeginDocument{%
\expandafter\renewcommand\csname r@tocindent0\endcsname{0pt}
}
\def\l@subsection{\@tocline{2}{0pt}{2.5pc}{5pc}{}}
\def\l@subsubsection{\@tocline{2}{0pt}{4.5pc}{5pc}{}}
\makeatother

\advance\footskip0.4cm
\textheight=54pc    %a4paper
\textheight=50.5pc %letterpaper
\advance\textheight-0.4cm
\calclayout

%Font settings
%\usepackage{anyfontsize}
%Footnote settings
%\usepackage{mathptmx}
\usepackage{footmisc}
%	\renewcommand*{\thefootnote}{\fnsymbol{footnote}}
\usepackage{commath}
%Further math environments
%Further math fonts (loads amsfonts implicitely)
\usepackage{amssymb}
%Redefinition of \text
%\usepackage{amstext}
\usepackage{upref}
%Graphics
%\usepackage{graphicx}
%\usepackage{caption}
%\usepackage{subcaption}
%Frames
\usepackage{mdframed}
\allowdisplaybreaks
%\usepackage{interval}
\newcommand{\toup}{%
  \mathrel{\nonscript\mkern-1.2mu\mkern1.2mu{\uparrow}}%
}
\newcommand{\todown}{%
  \mathrel{\nonscript\mkern-1.2mu\mkern1.2mu{\downarrow}}%
}
\AtBeginDocument{\renewcommand*\d{\mathop{}\!\mathrm{d}}}
\renewcommand{\Re}{\operatorname{Re}}
\renewcommand{\Im}{\operatorname{Im}}
\DeclareMathOperator\Log{Log}
\DeclareMathOperator\Arg{Arg}
\DeclareMathOperator\sech{sech}
\DeclareMathOperator*\esssup{ess.sup}
\DeclareMathOperator\id{id}
%\usepackage{hhline}
%\usepackage{booktabs} 
%\usepackage{array}
%\usepackage{xfrac} 
%\everymath{\displaystyle}
%Enumerate
\usepackage{tikz}
\usetikzlibrary{external}
\tikzexternalize % activate!
\usetikzlibrary{patterns}
\pgfdeclarepatternformonly{adjusted lines}{\pgfqpoint{-1pt}{-1pt}}{\pgfqpoint{40pt}{40pt}}{\pgfqpoint{39pt}{39pt}}%
{
  \pgfsetlinewidth{.8pt}
  \pgfpathmoveto{\pgfqpoint{0pt}{0pt}}
  \pgfpathlineto{\pgfqpoint{39.1pt}{39.1pt}}
  \pgfusepath{stroke}
}
\usepackage{enumitem} 
%\renewcommand{\labelitemi}{$\bullet$}
%\renewcommand{\labelitemii}{$\ast$}
%\renewcommand{\labelitemiii}{$\cdot$}
%\renewcommand{\labelitemiv}{$\circ$}
%Colors
%\usepackage{color}
%\usepackage[cmtip, all]{xy}
%Theorems
\newtheoremstyle{bold}              	 %Name
  {}                                     %Space above
  {}                                     %Space below
  {\itshape}		                     %Body font
  {}                                     %Indent amount
  {\bfseries}                             %Theorem head font
  {.}                                    %Punctuation after theorem head
  { }                                    %Space after theorem head, ' ', 
  										 %	or \newline
  {\thmname{#1}\thmnumber{ #2}\thmnote{ (#3)}} 
\theoremstyle{bold}
\newtheorem*{definition*}{Definition}
\newtheorem{definition}{Definition}[section]
\newtheorem*{lemma*}{Lemma}
\newtheorem{lemma}{Lemma}[section]
\newtheorem{Proof}{Proof}[section]
\newtheorem{proposition}{Proposition}[section]
\newtheorem{properties}{Properties}[section]
\newtheorem{corollary}{Corollary}[section]
\newtheorem*{theorem*}{Theorem}
\newtheorem{theorem}{Theorem}[section]
\newtheorem{example}{Example}[section]
\newtheoremstyle{nonitalic}            	 %Name
  {}                                     %Space above
  {}                                     %Space below
  {}				                     %Body font
  {}                                     %Indent amount
  {\bfseries}                             %Theorem head font
  {.}                                    %Punctuation after theorem head
  { }                                    %Space after theorem head, ' ', 
  										 %	or \newline
  {\thmname{#1}\thmnumber{ #2}\thmnote{ (#3)}}
\theoremstyle{nonitalic}
\newtheorem*{remark*}{Remark}
\newtheorem{remark}{Remark}[section]
%German non-ASCII-Characters
%Graphics-Tool
%\usepackage{tikz}
%\usepackage{tikzscale}
%\usepackage{bbm}
%\usepackage{bera}
%Listing-Setup
%Bibliographie
\usepackage[backend=bibtex, style=alphabetic]{biblatex}
%\usepackage[babel, german = swiss]{csquotes}
\bibliography{Bibliography}
%PDF-Linking
%\usepackage[hyphens]{url}
\usepackage[bookmarksopen=true,bookmarksnumbered=true]{hyperref}
%\PassOptionsToPackage{hyphens}{url}\usepackage{hyperref}
\hypersetup{
  colorlinks   = true, %Colours links instead of ugly boxes
  urlcolor     = blue, %Colour for external hyperlinks
  linkcolor    = blue, %Colour of internal links
  citecolor    = blue %Colour of citations
}
%Weierstrass-P symbol for power set
\newcommand{\powerset}{\raisebox{.15\baselineskip}{\Large\ensuremath{\wp}}}
\newcommand{\bld}[1]{\boldmath\textit{\textbf{#1}}\unboldmath}

\DeclareMathOperator{\Poi}{Poi}
\DeclareMathOperator{\Bin}{Bin}

\title{Solutions Sheet 4}
\author{Yannis B\"{a}hni}
\address[Yannis B\"{a}hni]{University of Zurich, R\"{a}mistrasse 71, 8006 Zurich}
\email[Yannis B\"{a}hni]{\href{mailto:yannis.baehni@uzh.ch}{yannis.baehni@uzh.ch}}

\begin{document}
\maketitle
\thispagestyle{fancy}
\begin{enumerate}[label = \textbf{Exercise \arabic*.},wide = 0pt, itemsep=1.5ex]
	\item To give the proof more structure, it is divided into three steps despite the structure of the exercise itself. I think this is more natural.
		\begin{lemma}
			$Q\mid_{\mathcal{A}_0} = P$.
		\end{lemma}

		\begin{proof}
			Let $A \in \mathcal{A}_0$. Consider the sequence $(B_n)_{n \in \mathbb{N}}$ in $\mathcal{A}_0$ defined by 
			\begin{align*}
				B_n := \begin{cases}
					A & n = 1\\
					\varnothing & n > 1
				\end{cases}.
			\end{align*}

			Clearly $A \subseteq \bigcup_{n \in \mathbb{N}}B_n$ and thus
			\begin{equation}
				Q(A) \leq \sum_{n \in \mathbb{N}} P(B_n) = P(A).
			\end{equation}

			Let $(B_n)_{n \in \mathbb{N}}$ be a sequence in $\mathcal{A}_0$ such that $A \subseteq \bigcup_{n \in \mathbb{N}} B_n$. Therefore $A = \bigcup_{n \in \mathbb{N}} (B_n \cap A)$ and thus by subadditivity of $P$
			\begin{equation}
				P(A) \leq \sum_{n \in \mathbb{N}} P(A_n \cap A) \leq \sum_{n \in \mathbb{N}} P(A_n).
			\end{equation}

			Thus $P(A) \leq Q(A)$ since the sequences were arbitrary.
		\end{proof}
		\begin{lemma}
			$Q: 2^\Omega \to \intcc{0,\infty}$ is an outer measure.
		\end{lemma}
		\begin{proof}
		Clearly $Q(\varnothing) = 0$ by the observation tha $\varnothing \subseteq \bigcup_{n \in \mathbb{N}} B_n$ where $B_n := \varnothing$ for all $n \in \mathbb{N}$ and that $P(\varnothing) = 0$ since $P$ is a probability. Observe that if $(B_n)_{n \in \mathbb{N}}$ is a sequence in $\mathcal{A}_0$ such that $B \subseteq \bigcup_{n \in \mathbb{N}}B_n$ we also have $A \subseteq \bigcup_{n \in \mathbb{N}}B_n$ since $A \subseteq B$. Hence the infimum in $Q(A)$ is taken on a large set than $Q(B)$, thus $Q(A) \leq Q(B)$. Let $(A_n)_{n \in \mathbb{N}}$ be a sequence in $2^\Omega$. For any $A \in 2^\Omega$ and $\varepsilon > 0$ we find by definition of $Q(A)$ a sequence $(B_n)_{n \in \mathbb{N}}$ in $\mathcal{A}_0$ such that $A \subseteq \bigcup_{n \in \mathbb{N}} B_n$ and 
		\begin{equation}
			Q(A) \leq \sum_{n \in \mathbb{N}} P(B_n) < Q(A) + \varepsilon.
		\end{equation}

		Thus for any $n \in \mathbb{N}$ we find a sequence $(B_{n,k})_{k \in \mathbb{N}}$ in $\mathcal{A}_0$ such that
		\begin{equation}
			\sum_{k \in \mathbb{N}} P(B_{n,k}) \leq Q(A_n) + \frac{\varepsilon}{2^n} \qquad n \in \mathbb{N}  
		\end{equation}

		\noindent and $A_n \subseteq \bigcup_{k \in \mathbb{N}} B_{n,k}$. Clearly $\bigcup_{n \in \mathbb{N}} A_n \subseteq \bigcup_{n \in \mathbb{N}}\bigcup_{k \in \mathbb{N}} B_{n,k}$ and so
		\begin{equation}
			Q\del[4]{\bigcup_{n \in \mathbb{N}} A_n} \leq \sum_{n \in \mathbb{N}}\sum_{k \in \mathbb{N}} P(B_{n,k}) \leq \sum\limits_{n \in \mathbb{N}}Q(A_n) + \varepsilon.
		\end{equation} 
	\end{proof}

	\begin{lemma}
		Each $B \in \mathcal{A}_0$ is $Q$-measurable.
	\end{lemma}

	\begin{proof}
		Let $A \subseteq \Omega$. Let $\varepsilon > 0$. In the same manner as in question $2$ we find a sequence $(A_n)_{n \in \mathbb{N}}$ in $\mathcal{A}_0$ such that 
		\begin{equation}
			\sum_{n \in \mathbb{N}} P(A_n) - \varepsilon \leq Q(A).
		\end{equation}

		Furthermore
		\begin{equation}
			Q(A) \geq \sum_{n \in \mathbb{N}}P(A_n \cap B) + \sum_{n \in \mathbb{N}} P(A_n \cap B^c) - \varepsilon \geq Q(A \cap B) + Q(A \cap B^c) - \varepsilon
		\end{equation}

		\noindent by the additivity of $P$.

	\end{proof}
	\item 

	\item 
		\begin{lemma}
			Let $F: \mathbb{R} \to \mathbb{R}$ be a distribution function. Then
			\begin{equation}
				P\intoo{a,b} = F(b-) - F(a) \qquad \text{and} \qquad P\intco{a,b} = F(b- ) - F(a-).
			\end{equation}
			\noindent holds for all $-\infty \leq a < b \leq \infty$ in the first case and $-\infty < a < b \leq \infty$ in the second.
			\label{lem:3}
		\end{lemma}

		\begin{proof}
			Note that 
			\begin{equation}
				\intoo{a,b} = \bigcup_{n \in \mathbb{N}_{>0}} \intoc{-\infty,b - 1/n} \setminus \intoc{-\infty,a}.
				\label{eq:union}
			\end{equation}

			By (\ref{eq:union}) and the fact that $F$ induces a probability measure $P$ we have 
			\begin{align*}
				P\intoo{a,b} &= P\del[4]{\bigcup_{n \in \mathbb{N}_{>0}} \intoc{-\infty,b - 1/n} \setminus \intoc{-\infty,a}}\\
				&= P\del[4]{\bigcup_{n \in \mathbb{N}_{>0}} \intoc{-\infty,b - 1/n}} - P\intoc{-\infty,a}\\
				&= \lim_{n \to \infty} P\intoc{-\infty,b - 1/n} - P\intoc{-\infty,a}\\
				&= \lim_{n \to \infty} F(b - 1/n) - F(a)\\
				&= F(b-) - F(a).
			\end{align*}

			The second statement follows similarly from
			\begin{equation}
				\intco{a,b} = \bigcup_{n \in \mathbb{N}_{>0}} \intoc{-\infty,b - 1/n} \setminus \bigcup_{n \in \mathbb{N}_{>0}}\intoc{-\infty,a - 1/n}
			\end{equation}
		\end{proof}
		
		\begin{remark}
			Note that for any $x \in \mathbb{R}$ the limit $F(x-)$ exists since $F$ is monotone increasing and bounded from above by $1$. Indeed, towards a contradiction assume that there exists some $x_0 \in \mathbb{R}$ such that $F(x_0) > 1$. Hence $F(x_0) - 1 > 0$ and so there exsist $M \in \mathbb{R}$ such that $F(x) < F(x_0)$ for all $x > M$. Contradiction.
		\end{remark}
		$F: \mathbb{R} \to \mathbb{R}$ can be rewritten as 
		\begin{equation}
			F(x) = \frac{1}{4}\chi_{\intco{0,1}}(x) + \frac{3}{4}\chi_{\intco{1,2}}(x) + \chi_{\intco{2,\infty}}(x).
		\end{equation}
	
		Clearly $F$ is monotone increasing, continuous on the right and
		\begin{equation}
			\lim_{x \to -\infty} F(x) = 0 \qquad \text{and} \qquad \lim_{x \to \infty} F(x) = 1.
		\end{equation}

		Thus there exists a unique probability measure $P$ on $(\mathbb{R},\mathcal{B}(\mathbb{R}))$ such that 
		\begin{equation}
			F\intoc{a,b} = F(b) - F(a)
		\end{equation}
		\noindent for all $-\infty \leq a < b < \infty$. From lemma \ref{lem:3} immediately follows
		\begin{equation*}
			P(A) = 1 \qquad  P(B) = 1 \qquad  P(C) = 0 \qquad  P(D) = \frac{3}{4} \qquad P(E) = 0.
		\end{equation*}
		
		\item
			~
			\begin{enumerate}[label = \arabic*.,wide = 0pt, itemsep=1.5ex]
				\item An example would be the function $F: \mathbb{R} \to \mathbb{R}$ defined by
					\begin{equation}
						F(x) := \sum_{n = 1}^\infty (1 - 1/n) \chi_{\intco{n-1,n}}(x).
					\end{equation}

					$F$ is discontinuous at any $n \in \mathbb{N}_{>0}$.
			\end{enumerate}

\end{enumerate}
%\originalsectionstyle
\printbibliography
\end{document}
