\input{header.tex}
\DeclareMathOperator{\Poi}{Poi}
\DeclareMathOperator{\Bin}{Bin}

\title{Solutions Sheet 4}
\author{Yannis B\"{a}hni}
\address[Yannis B\"{a}hni]{University of Zurich, R\"{a}mistrasse 71, 8006 Zurich}
\email[Yannis B\"{a}hni]{\href{mailto:yannis.baehni@uzh.ch}{yannis.baehni@uzh.ch}}

\begin{document}
\maketitle
\thispagestyle{fancy}
\begin{enumerate}[label = \textbf{Exercise \arabic*.},wide = 0pt, itemsep=1.5ex]
	\item To give the proof more structure, it is divided into three steps despite the structure of the exercise itself. I think this is more natural.
		\begin{lemma}
			$Q\mid_{\mathcal{A}_0} = P$.
		\end{lemma}

		\begin{proof}
			Let $A \in \mathcal{A}_0$. Consider the sequence $(B_n)_{n \in \mathbb{N}}$ in $\mathcal{A}_0$ defined by 
			\begin{align*}
				B_n := \begin{cases}
					A & n = 1\\
					\varnothing & n > 1
				\end{cases}.
			\end{align*}

			Clearly $A \subseteq \bigcup_{n \in \mathbb{N}}B_n$ and thus
			\begin{equation}
				Q(A) \leq \sum_{n \in \mathbb{N}} P(B_n) = P(A).
			\end{equation}

			Let $(B_n)_{n \in \mathbb{N}}$ be a sequence in $\mathcal{A}_0$ such that $A \subseteq \bigcup_{n \in \mathbb{N}} B_n$. Therefore $A = \bigcup_{n \in \mathbb{N}} (B_n \cap A)$ and thus by subadditivity of $P$
			\begin{equation}
				P(A) \leq \sum_{n \in \mathbb{N}} P(A_n \cap A) \leq \sum_{n \in \mathbb{N}} P(A_n).
			\end{equation}

			Thus $P(A) \leq Q(A)$ since the sequences were arbitrary.
		\end{proof}
		\begin{lemma}
			$Q: 2^\Omega \to \intcc{0,\infty}$ is an outer measure.
		\end{lemma}
		\begin{proof}
		Clearly $Q(\varnothing) = 0$ by the observation tha $\varnothing \subseteq \bigcup_{n \in \mathbb{N}} B_n$ where $B_n := \varnothing$ for all $n \in \mathbb{N}$ and that $P(\varnothing) = 0$ since $P$ is a probability. Observe that if $(B_n)_{n \in \mathbb{N}}$ is a sequence in $\mathcal{A}_0$ such that $B \subseteq \bigcup_{n \in \mathbb{N}}B_n$ we also have $A \subseteq \bigcup_{n \in \mathbb{N}}B_n$ since $A \subseteq B$. Hence the infimum in $Q(A)$ is taken on a large set than $Q(B)$, thus $Q(A) \leq Q(B)$. Let $(A_n)_{n \in \mathbb{N}}$ be a sequence in $2^\Omega$. For any $A \in 2^\Omega$ and $\varepsilon > 0$ we find by definition of $Q(A)$ a sequence $(B_n)_{n \in \mathbb{N}}$ in $\mathcal{A}_0$ such that $A \subseteq \bigcup_{n \in \mathbb{N}} B_n$ and 
		\begin{equation}
			Q(A) \leq \sum_{n \in \mathbb{N}} P(B_n) < Q(A) + \varepsilon.
		\end{equation}

		Thus for any $n \in \mathbb{N}$ we find a sequence $(B_{n,k})_{k \in \mathbb{N}}$ in $\mathcal{A}_0$ such that
		\begin{equation}
			\sum_{k \in \mathbb{N}} P(B_{n,k}) \leq Q(A_n) + \frac{\varepsilon}{2^n} \qquad n \in \mathbb{N}  
		\end{equation}

		\noindent and $A_n \subseteq \bigcup_{k \in \mathbb{N}} B_{n,k}$. Clearly $\bigcup_{n \in \mathbb{N}} A_n \subseteq \bigcup_{n \in \mathbb{N}}\bigcup_{k \in \mathbb{N}} B_{n,k}$ and so
		\begin{equation}
			Q\del[4]{\bigcup_{n \in \mathbb{N}} A_n} \leq \sum_{n \in \mathbb{N}}\sum_{k \in \mathbb{N}} P(B_{n,k}) \leq \sum\limits_{n \in \mathbb{N}}Q(A_n) + \varepsilon.
		\end{equation} 
	\end{proof}

	\begin{lemma}
		Each $B \in \mathcal{A}_0$ is $Q$-measurable.
	\end{lemma}

	\begin{proof}
		
	\end{proof}
	\item 

	\item 
		\begin{lemma}
			Let $F: \mathbb{R} \to \mathbb{R}$ be a distribution function. Then
			\begin{equation}
				P\intoo{a,b} = F(b-) - F(a) \qquad \text{and} \qquad P\intco{a,b} = F(b- ) - F(a-).
			\end{equation}
			\noindent holds for all $-\infty \leq a < b \leq \infty$.
			\label{lem:3}
		\end{lemma}

		\begin{proof}
			Note that 
			\begin{equation}
				\intoo{a,b} = \bigcup_{n \in \mathbb{N}_{>0}} \intoc{-\infty,b - 1/n} \setminus \intoc{-\infty,a}.
				\label{eq:union}
			\end{equation}

			By (\ref{eq:union}) and the fact that $F$ induces a probability measure $P$ we have 
			\begin{align*}
				P\intoo{a,b} &= P\del[4]{\bigcup_{n \in \mathbb{N}_{>0}} \intoc{-\infty,b - 1/n} \setminus \intoc{-\infty,a}}\\
				&= P\del[4]{\bigcup_{n \in \mathbb{N}_{>0}} \intoc{-\infty,b - 1/n}} - P\intoc{-\infty,a}\\
				&= \lim_{n \to \infty} P\intoc{-\infty,b - 1/n} - P\intoc{-\infty,a}\\
				&= \lim_{n \to \infty} F(b - 1/n) - F(a)\\
				&= F(b-) - F(a).
			\end{align*}

			The second statement follows similarly from
			\begin{equation}
				\intco{a,b} = \bigcup_{n \in \mathbb{N}_{>0}} \intoc{-\infty,b - 1/n} \setminus \bigcup_{n \in \mathbb{N}_{>0}}\intoc{-\infty,a - 1/n}
			\end{equation}
		\end{proof}
		
		\begin{remark}
			Note that for any $x \in \mathbb{R}$ the limit $F(x-)$ exists since $F$ is monotone increasing and bounded from above by $1$. Indeed, towards a contradiction assume that there exists some $x_0 \in \mathbb{R}$ such that $F(x_0) > 1$. Hence $F(x_0) - 1 > 0$ and so there exsist $M \in \mathbb{R}$ such that $F(x) < F(x_0)$ for all $x > M$. Contradiction.
		\end{remark}
		$F: \mathbb{R} \to \mathbb{R}$ can be rewritten as 
		\begin{equation}
			F(x) = \frac{1}{4}\chi_{\intco{0,1}}(x) + \frac{3}{4}\chi_{\intco{1,2}}(x) + \chi_{\intco{2,\infty}}(x).
		\end{equation}
	
		Clearly $F$ is monotone increasing, continuous on the right and
		\begin{equation}
			\lim_{x \to -\infty} F(x) = 0 \qquad \text{and} \qquad \lim_{x \to \infty} F(x) = 1.
		\end{equation}

		Thus there exists a unique probability measure $P$ on $(\mathbb{R},\mathcal{B}(\mathbb{R}))$ such that 
		\begin{equation}
			F\intoc{a,b} = F(b) - F(a)
		\end{equation}
		\noindent for all $-\infty \leq a < b < \infty$. From lemma \ref{lem:3} immediately follows
		\begin{equation*}
			P(A) = 1 \qquad  P(B) = 1 \qquad  P(C) = 0 \qquad  P(D) = \frac{3}{4} \qquad P(E) = 0.
		\end{equation*}
		
		\item
			~
			\begin{enumerate}[label = \arabic*.,wide = 0pt, itemsep=1.5ex]
				\item An example would be the function $F: \mathbb{R} \to \mathbb{R}$ defined by
					\begin{equation}
						F(x) := \sum_{n = 1}^\infty (1 - 1/n) \chi_{\intco{n-1,n}}(x).
					\end{equation}

					$F$ is discontinuous at any $n \in \mathbb{N}_{>0}$.
			\end{enumerate}

\end{enumerate}
%\originalsectionstyle
\printbibliography
\end{document}
