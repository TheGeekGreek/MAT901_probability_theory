%%%%%%%%%%%%%%%%%%%%%%%%%%%%%%%%%%%%%%%%%%%%%%%%%%%%%%%%%%%%%%%%%%%%%%%%%%
%Author:																 %
%-------																 %
%Yannis Baehni at University of Zurich									 %
%baehni.yannis@uzh.ch													 %
%																		 %
%Version log:															 %
%------------															 %
%06/02/16 . Basic structure												 %
%04/08/16 . Layout changes including section, contents, abstract.		 %
%%%%%%%%%%%%%%%%%%%%%%%%%%%%%%%%%%%%%%%%%%%%%%%%%%%%%%%%%%%%%%%%%%%%%%%%%%

%Page Setup
\documentclass[
	11pt, 
	oneside, 
	a4paper,
	reqno,
	final
]{amsart}

\usepackage[
	left = 3cm, 
	right = 3cm, 
	top = 3cm, 
	bottom = 3cm
]{geometry}

%Headers and footers
\usepackage{fancyhdr}
	\pagestyle{fancy}
	%Clear fields
	\fancyhf{}
	%Header right
	\fancyhead[R]{
		\footnotesize
		Yannis B\"{a}hni\\
		\href{mailto:yannis.baehni@uzh.ch}{yannis.baehni@uzh.ch}
	}
	%Header left
	\fancyhead[L]{
		\footnotesize
		MAT901: Stochastics I\\
		Spring Semester 2017
	}
	%Page numbering in footer
	\fancyfoot[C]{\thepage}
	%Separation line header and footer
	\renewcommand{\headrulewidth}{0.4pt}
	%\renewcommand{\footrulewidth}{0.4pt}
	
	\setlength{\headheight}{19pt} 

%Title
\usepackage[foot]{amsaddr}
\usepackage{mathrsfs}
%\usepackage{mathptmx}
\usepackage{xspace}
\makeatletter
\def\@textbottom{\vskip \z@ \@plus 1pt}
\let\@texttop\relax
\usepackage{etoolbox}
\patchcmd{\abstract}{\scshape\abstractname}{\textbf{\abstractname}}{}{}

\usepackage[all,cmtip]{xy}

%Switching commands for different section formats
%Mainsectionsytle
\newcommand{\mainsectionstyle}{%
  	\renewcommand{\@secnumfont}{\bfseries}
  	\renewcommand\section{\@startsection{section}{1}%
    	\z@{.5\linespacing\@plus.7\linespacing}{-.5em}%
    	{\normalfont\bfseries}}%
	\renewcommand\subsection{\@startsection{subsection}{2}%
    	\z@{.5\linespacing\@plus.7\linespacing}{-.5em}%
    	{\normalfont\bfseries}}%
	\renewcommand\subsubsection{\@startsection{subsubsection}{3}%
    	\z@{.5\linespacing\@plus.7\linespacing}{-.5em}%
    	{\normalfont\bfseries}}%
}
\newcommand{\originalsectionstyle}{%
\def\@secnumfont{\bfseries}%\mdseries
\def\section{\@startsection{section}{1}%
  \z@{.7\linespacing\@plus\linespacing}{.5\linespacing}%
  {\normalfont\bfseries\centering}}
}
%Formatting title of TOC
\renewcommand{\contentsnamefont}{\bfseries}
%Table of Contents
\setcounter{tocdepth}{3}

% Add bold to \section titles in ToC and remove . after numbers
\renewcommand{\tocsection}[3]{%
  \indentlabel{\@ifnotempty{#2}{\bfseries\ignorespaces#1 #2\quad}}\bfseries#3}
% Remove . after numbers in \subsection
\renewcommand{\tocsubsection}[3]{%
  \indentlabel{\@ifnotempty{#2}{\ignorespaces#1 #2\quad}}#3}
\let\tocsubsubsection\tocsubsection% Update for \subsubsection
%...

\newcommand\@dotsep{4.5}
\def\@tocline#1#2#3#4#5#6#7{\relax
  \ifnum #1>\c@tocdepth % then omit
  \else
    \par \addpenalty\@secpenalty\addvspace{#2}%
    \begingroup \hyphenpenalty\@M
    \@ifempty{#4}{%
      \@tempdima\csname r@tocindent\number#1\endcsname\relax
    }{%
      \@tempdima#4\relax
    }%
    \parindent\z@ \leftskip#3\relax \advance\leftskip\@tempdima\relax
    \rightskip\@pnumwidth plus1em \parfillskip-\@pnumwidth
    #5\leavevmode\hskip-\@tempdima{#6}\nobreak
    \leaders\hbox{$\m@th\mkern \@dotsep mu\hbox{.}\mkern \@dotsep mu$}\hfill
    \nobreak
    \hbox to\@pnumwidth{\@tocpagenum{\ifnum#1=1\bfseries\fi#7}}\par% <-- \bfseries for \section page
    \nobreak
    \endgroup
  \fi}
\AtBeginDocument{%
\expandafter\renewcommand\csname r@tocindent0\endcsname{0pt}
}
\def\l@subsection{\@tocline{2}{0pt}{2.5pc}{5pc}{}}
\def\l@subsubsection{\@tocline{2}{0pt}{4.5pc}{5pc}{}}
\makeatother

\advance\footskip0.4cm
\textheight=54pc    %a4paper
\textheight=50.5pc %letterpaper
\advance\textheight-0.4cm
\calclayout

%Font settings
%\usepackage{anyfontsize}
%Footnote settings
%\usepackage{mathptmx}
\usepackage{footmisc}
%	\renewcommand*{\thefootnote}{\fnsymbol{footnote}}
\usepackage{commath}
%Further math environments
%Further math fonts (loads amsfonts implicitely)
\usepackage{amssymb}
%Redefinition of \text
%\usepackage{amstext}
\usepackage{upref}
%Graphics
%\usepackage{graphicx}
%\usepackage{caption}
%\usepackage{subcaption}
%Frames
\usepackage{mdframed}
\allowdisplaybreaks
%\usepackage{interval}
\newcommand{\toup}{%
  \mathrel{\nonscript\mkern-1.2mu\mkern1.2mu{\uparrow}}%
}
\newcommand{\todown}{%
  \mathrel{\nonscript\mkern-1.2mu\mkern1.2mu{\downarrow}}%
}
\AtBeginDocument{\renewcommand*\d{\mathop{}\!\mathrm{d}}}
\renewcommand{\Re}{\operatorname{Re}}
\renewcommand{\Im}{\operatorname{Im}}
\DeclareMathOperator\Log{Log}
\DeclareMathOperator\Arg{Arg}
\DeclareMathOperator\sech{sech}
\DeclareMathOperator*\esssup{ess.sup}
\DeclareMathOperator\id{id}
%\usepackage{hhline}
%\usepackage{booktabs} 
%\usepackage{array}
%\usepackage{xfrac} 
%\everymath{\displaystyle}
%Enumerate
\usepackage{tikz}
\usetikzlibrary{external}
\tikzexternalize % activate!
\usetikzlibrary{patterns}
\pgfdeclarepatternformonly{adjusted lines}{\pgfqpoint{-1pt}{-1pt}}{\pgfqpoint{40pt}{40pt}}{\pgfqpoint{39pt}{39pt}}%
{
  \pgfsetlinewidth{.8pt}
  \pgfpathmoveto{\pgfqpoint{0pt}{0pt}}
  \pgfpathlineto{\pgfqpoint{39.1pt}{39.1pt}}
  \pgfusepath{stroke}
}
\usepackage{enumitem} 
%\renewcommand{\labelitemi}{$\bullet$}
%\renewcommand{\labelitemii}{$\ast$}
%\renewcommand{\labelitemiii}{$\cdot$}
%\renewcommand{\labelitemiv}{$\circ$}
%Colors
%\usepackage{color}
%\usepackage[cmtip, all]{xy}
%Theorems
\newtheoremstyle{bold}              	 %Name
  {}                                     %Space above
  {}                                     %Space below
  {\itshape}		                     %Body font
  {}                                     %Indent amount
  {\bfseries}                             %Theorem head font
  {.}                                    %Punctuation after theorem head
  { }                                    %Space after theorem head, ' ', 
  										 %	or \newline
  {\thmname{#1}\thmnumber{ #2}\thmnote{ (#3)}} 
\theoremstyle{bold}
\newtheorem*{definition*}{Definition}
\newtheorem{definition}{Definition}[section]
\newtheorem*{lemma*}{Lemma}
\newtheorem{lemma}{Lemma}[section]
\newtheorem{Proof}{Proof}[section]
\newtheorem{proposition}{Proposition}[section]
\newtheorem{properties}{Properties}[section]
\newtheorem{corollary}{Corollary}[section]
\newtheorem*{theorem*}{Theorem}
\newtheorem{theorem}{Theorem}[section]
\newtheorem{example}{Example}[section]
\newtheoremstyle{nonitalic}            	 %Name
  {}                                     %Space above
  {}                                     %Space below
  {}				                     %Body font
  {}                                     %Indent amount
  {\bfseries}                             %Theorem head font
  {.}                                    %Punctuation after theorem head
  { }                                    %Space after theorem head, ' ', 
  										 %	or \newline
  {\thmname{#1}\thmnumber{ #2}\thmnote{ (#3)}}
\theoremstyle{nonitalic}
\newtheorem*{remark*}{Remark}
\newtheorem{remark}{Remark}[section]
%German non-ASCII-Characters
%Graphics-Tool
%\usepackage{tikz}
%\usepackage{tikzscale}
%\usepackage{bbm}
%\usepackage{bera}
%Listing-Setup
%Bibliographie
\usepackage[backend=bibtex, style=alphabetic]{biblatex}
%\usepackage[babel, german = swiss]{csquotes}
\bibliography{Bibliography}
%PDF-Linking
%\usepackage[hyphens]{url}
\usepackage[bookmarksopen=true,bookmarksnumbered=true]{hyperref}
%\PassOptionsToPackage{hyphens}{url}\usepackage{hyperref}
\hypersetup{
  colorlinks   = true, %Colours links instead of ugly boxes
  urlcolor     = blue, %Colour for external hyperlinks
  linkcolor    = blue, %Colour of internal links
  citecolor    = blue %Colour of citations
}
%Weierstrass-P symbol for power set
\newcommand{\powerset}{\raisebox{.15\baselineskip}{\Large\ensuremath{\wp}}}
\newcommand{\bld}[1]{\boldmath\textit{\textbf{#1}}\unboldmath}


\title{Solutions Sheet 2}
\author{Yannis B\"{a}hni}
\address[Yannis B\"{a}hni]{University of Zurich, R\"{a}mistrasse 71, 8006 Zurich}
\email[Yannis B\"{a}hni]{\href{mailto:yannis.baehni@uzh.ch}{yannis.baehni@uzh.ch}}

\begin{document}
\maketitle
\thispagestyle{fancy}
\begin{enumerate}[label = \textbf{Exercise \arabic*.},wide = 0pt, itemsep=1.5ex]
	\item Let $(\Omega,\mathcal{A},P)$ be a probability space. Recall, that for $A \in \mathcal{A}$ with $P(A) > 0$ the \emph{conditional probability of $B$ with respect to $A$} is defined by
		\begin{equation}
			P(B \vert A) := \frac{P(B \cap A)}{P(A)}.
		\end{equation}
		\begin{enumerate}[label = \arabic*.,wide = 0pt, itemsep=1.5ex]
			\item Let $A_1,A_2 \in \mathcal{A}$ with $0 < P(A_2) < 1$. Observe, that by $0 < P(A_2) < 1$ the conditional probability $P(B\vert A_2^c)$ is well-defined since $P(A_2^c) = 1 - P(A_2) > 0$. Thus
				\begin{align*}
					P(A_1) &= P((A_1 \cap A_2) \cup (A_1 \cap A_2^c))\\
					&= P(A_1 \cap A_2) + P(A_1 \cap A_2^c)\\
					&= \frac{P(A_1 \cap A_2)}{P(A_2)}P(A_2) + \frac{P(A_1 \cap A_2^c)}{P(A_2^c)}P(A_2^c)\\
					&= P(A_1 \vert A_2)P(A_2) + P(A_1 \vert A_2^c)P(A_2^c).
				\end{align*}

			\item We simply have 
				\begin{equation*}
					P(A_3 \vert A_1 \cap A_2) = \frac{P(A_3 \cap A_1 \cap A_2)}{P(A_1 \cap A_2)} = \frac{P(A_3)P(A_1)P(A_2)}{P(A_1)P(A_2)} = P(A_3).
				\end{equation*}

				\noindent from the definition of independence.

			\item First we prove two auxiliary results. 
				\begin{lemma}
					Let $A_1, \dots,A_n \in \mathcal{A}$ be independent. Then $A_1^c, \dots, A_n^c$ are also independent.
					\label{lem:independence_complements}
				\end{lemma}

				\begin{proof}
					It is enough to consider the case $A_1,\dots,A_{i - 1},A_i^c,A_{i + 1},\dots,A_n$ for some $i \in \cbr[0]{1,\dots,n}$. Let $I \subseteq \cbr[0]{1,\dots,n}$. If $i \notin I$, there is nothing to prove. So assume $i \in I$. Then we have
					\begin{align*}
						P\del[4]{A_i^c \cap \bigcap_{\iota \in I \setminus \cbr[0]{i}} A_\iota} &= P\del[4]{\bigcap_{\iota \in I \setminus \cbr[0]{i}}A_\iota}- P\del[4]{\bigcap_{\iota \in I} A_\iota}\\
						&= \prod_{\iota \in I \setminus \cbr[0]{i}} P(A_\iota) - \prod_{\iota \in I} P(A_\iota)\\
						&= (1 - P(A_i))\prod_{\iota \in I \setminus \cbr[0]{i}} P(A_\iota)\\
						&= P(A_i^c) \prod_{\iota \in I \setminus \cbr[0]{i}} P(A_\iota).
					\end{align*}
				\end{proof}

				\begin{lemma}
					Let $f: \intco{0,1} \to \mathbb{R}$ be defined by $f(x) := \log(1-x)+x$. Then $f \leq 0$.
					\label{lem:fun}
				\end{lemma}

				\begin{proof}
					$f$ is clearly differentiable on $\intco{0,1}$ with
					\begin{equation}
						f'(x) = 1 - \frac{1}{1 - x} = \frac{x}{x - 1} \leq 0.
					\end{equation}

					Hence $f$ is monotonically decreasing on $\intco{0,1}$. By $f(0) = 0$ we conclude $f \leq 0$.
				\end{proof}
				If $P(A_i) = 1$ for some $i \in \cbr[0]{1,\dots,n}$ we have
				\begin{equation*}
					P\del[4]{\del[4]{\bigcup_{j = 1}^n A_j}^c} \leq P(A_i^c) = 0 \leq \exp\del[4]{-\sum_{j = 1}^n P(A_j)}
				\end{equation*}

				\noindent since $A_i \subseteq \bigcup_{j = 1}^n A_j$. Therefore it is enough to consider $P(A_i) < 1$ for $i = 1,\dots,n$. Using lemma \ref{lem:independence_complements} and \ref{lem:fun} we get
				\begin{align*}
					P\del[4]{\del[4]{\bigcup_{i = 1}^n A_i}^c} &= P\del[4]{\bigcap_{i = 1}^n A_i^c}\\
					&= \prod_{i = 1}^n P(A_i^c)\\
					&= \exp \del[4]{\log \del[4]{\prod_{i = 1}^n P(A_i^c)}}\\
					&= \exp \del[4]{\sum_{i = 1}^n \log(P(A_i^c))}\\
					&= \exp \del[4]{\sum_{i = 1}^n \log(1 - P(A_i))}\\
					&\leq \exp \del[4]{-\sum_{i = 1}^n P(A_i)}.
				\end{align*}
		\end{enumerate}
\end{enumerate}
%\originalsectionstyle
\printbibliography
\end{document}
