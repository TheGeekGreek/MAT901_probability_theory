\input{header.tex}

\title{Solutions Sheet 2}
\author{Yannis B\"{a}hni}
\address[Yannis B\"{a}hni]{University of Zurich, R\"{a}mistrasse 71, 8006 Zurich}
\email[Yannis B\"{a}hni]{\href{mailto:yannis.baehni@uzh.ch}{yannis.baehni@uzh.ch}}

\begin{document}
\maketitle
\thispagestyle{fancy}
\begin{enumerate}[label = \textbf{Exercise \arabic*.},wide = 0pt, itemsep=1.5ex]
	\item Let $(\Omega,\mathcal{A},P)$ be a probability space. Recall, that for $A \in \mathcal{A}$ with $P(A) > 0$ the \emph{conditional probability of $B$ with respect to $A$} is defined by
		\begin{equation}
			P(B \vert A) := \frac{P(B \cap A)}{P(A)}.
		\end{equation}
		\begin{enumerate}[label = \arabic*.,wide = 0pt, itemsep=1.5ex]
			\item Let $A_1,A_2 \in \mathcal{A}$ with $0 < P(A_2) < 1$. Observe, that by $0 < P(A_2) < 1$ the conditional probability $P(B\vert A_2^c)$ is well-defined since $P(A_2^c) = 1 - P(A_2) > 0$. Thus
				\begin{align*}
					P(A_1) &= P((A_1 \cap A_2) \cup (A_1 \cap A_2^c))\\
					&= P(A_1 \cap A_2) + P(A_1 \cap A_2^c)\\
					&= \frac{P(A_1 \cap A_2)}{P(A_2)}P(A_2) + \frac{P(A_1 \cap A_2^c)}{P(A_2^c)}P(A_2^c)\\
					&= P(A_1 \vert A_2)P(A_2) + P(A_1 \vert A_2^c)P(A_2^c).
				\end{align*}

			\item We have 
				\begin{equation*}
					P(A_3 \vert A_1 \cap A_2) = \frac{P(A_3 \cap A_1 \cap A_2)}{P(A_1 \cap A_2)} = \frac{P(A_3)P(A_1)P(A_2)}{P(A_1)P(A_2)} = P(A_3).
				\end{equation*}

			\item First we prove two auxiliary results. 
				\begin{lemma}
					Let $A_1, \dots,A_n \in \mathcal{A}$ be independent. Then $A_1^c, \dots, A_n^c$ are independent.
					\label{lem:independence_complements}
				\end{lemma}

				\begin{proof}
					It is enough to consider the case $A_1,\dots,A_{i - 1},A_i^c,A_{i + 1},\dots,A_n$ for some $i \in \cbr[0]{1,\dots,n}$. Let $I \subseteq \cbr[0]{1,\dots,n}$. If $i \notin I$, there is nothing to prove. So assume $i \in I$. Then we have
					\begin{align*}
						P\del[4]{A_i^c \cap \bigcap_{\iota \in I \setminus \cbr[0]{i}} A_\iota} &= P\del[4]{\bigcap_{\iota \in I \setminus \cbr[0]{i}}A_\iota}- P\del[4]{\bigcap_{\iota \in I} A_\iota}\\
						&= \prod_{\iota \in I \setminus \cbr[0]{i}} P(A_\iota) - \prod_{\iota \in I} P(A_\iota)\\
						&= (1 - P(A_i))\prod_{\iota \in I \setminus \cbr[0]{i}} P(A_\iota)\\
						&= P(A_i^c) \prod_{\iota \in I \setminus \cbr[0]{i}} P(A_\iota).
					\end{align*}
				\end{proof}

				\begin{lemma}
					Let $f: \intco{0,1} \to \mathbb{R}$ be defined by $f(x) := \log(1-x)+x$. Then $f \leq 0$.
					\label{lem:fun}
				\end{lemma}

				\begin{proof}
					$f$ is clearly differentiable on $\intco{0,1}$ with
					\begin{equation}
						f'(x) = 1 - \frac{1}{1 - x} = \frac{x}{x - 1} \leq 0.
					\end{equation}

					Hence $f$ is monotonically decreasing on $\intco{0,1}$. By $f(0) = 0$ we conclude $f \leq 0$.
				\end{proof}
				If $P(A_i) = 1$ for some $i \in \cbr[0]{1,\dots,n}$ we have
				\begin{equation*}
					P\del[4]{\del[4]{\bigcup_{j = 1}^n A_j}^c} \leq P(A_i^c) = 0 \leq \exp\del[4]{-\sum_{j = 1}^n P(A_j)}
				\end{equation*}

				\noindent since $A_i \subseteq \bigcup_{j = 1}^n A_j$. Therefore it is enough to consider $P(A_i) < 1$ for $i = 1,\dots,n$. Using lemma \ref{lem:independence_complements} and \ref{lem:fun} we get
				\begin{align*}
					P\del[4]{\del[4]{\bigcup_{i = 1}^n A_i}^c} &= P\del[4]{\bigcap_{i = 1}^n A_i^c}\\
					&= \prod_{i = 1}^n P(A_i^c)\\
					&= \exp \del[4]{\log \del[4]{\prod_{i = 1}^n P(A_i^c)}}\\
					&= \exp \del[4]{\sum_{i = 1}^n \log(P(A_i^c))}\\
					&= \exp \del[4]{\sum_{i = 1}^n \log(1 - P(A_i))}\\
					&\leq \exp \del[4]{-\sum_{i = 1}^n P(A_i)}
				\end{align*}
		\end{enumerate}
\end{enumerate}
%\originalsectionstyle
\printbibliography
\end{document}
